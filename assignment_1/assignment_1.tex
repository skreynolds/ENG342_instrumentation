%%%%%%%%%%%%%%%%%%%%%%%%%%%%%%%%%%%%%%%%%
% University/School Laboratory Report
% LaTeX Template
% Version 3.1 (25/3/14)
%
% This template has been downloaded from:
% http://www.LaTeXTemplates.com
%
% Original author:
% Linux and Unix Users Group at Virginia Tech Wiki 
% (https://vtluug.org/wiki/Example_LaTeX_chem_lab_report)
%
% License:
% CC BY-NC-SA 3.0 (http://creativecommons.org/licenses/by-nc-sa/3.0/)
%
%%%%%%%%%%%%%%%%%%%%%%%%%%%%%%%%%%%%%%%%%

%----------------------------------------------------------------------------------------
%	PACKAGES AND DOCUMENT CONFIGURATIONS
%----------------------------------------------------------------------------------------

\documentclass{article}

\usepackage[version=3]{mhchem} % Package for chemical equation typesetting
\usepackage{siunitx} % Provides the \SI{}{} and \si{} command for typesetting SI units
\usepackage{graphicx} % Required for the inclusion of images
\usepackage{natbib} % Required to change bibliography style to APA
\usepackage{amsmath} % Required for some math elements 
\usepackage{enumerate} % Required for the enumerate function
\usepackage[siunitx]{circuitikz} % Required for the drawing of circuit diagrams
\usepackage{caption}
\usepackage{graphicx}
\usepackage{subcaption}
\usepackage{xfrac}
\usepackage{float}
\usepackage{enumitem}
\usepackage{chemgreek}
\usepackage{pgfplots}
\usepackage[margin=0.5in]{geometry}

\setlength\parindent{0pt} % Removes all indentation from paragraphs

\renewcommand{\labelenumi}{\alph{enumi}.} % Make numbering in the enumerate environment by letter rather than number (e.g. section 6)

%\usepackage{times} % Uncomment to use the Times New Roman font

%----------------------------------------------------------------------------------------
%	DOCUMENT INFORMATION
%----------------------------------------------------------------------------------------

\title{Analogue Devices \\ Experiment 1 \\ ENG471} % Title

\author{Shane \textsc{Reynolds}} % Author name

\date{\today} % Date for the report

\begin{document}

\maketitle % Insert the title, author and date

\begin{center}
\begin{tabular}{l r}
Date Performed: & April 7, 2016 \\ % Date the experiment was performed
Instructor: & Dr Sina Vafi % Instructor/supervisor
\end{tabular}
\end{center}

% If you wish to include an abstract, uncomment the lines below
% \begin{abstract}
% Abstract text
% \end{abstract}

%----------------------------------------------------------------------------------------
%	SECTION 1
%----------------------------------------------------------------------------------------

\section{Objective}

An ideal tank heating process scenario will be mathematically modelled. Comparison the model will compared to simulated results, and discussed.

%----------------------------------------------------------------------------------------
%	SECTION 2
%----------------------------------------------------------------------------------------

\section{Background \& Model Derivation}
Consider the tank shown in Figure 1. Assuming that there is no heat transfer through the walls of the tank the \textbf{LAW OF WHAT} suggests that:
\begin{align*}
\Bigg[\parbox{3cm}{Rate of heat from incoming liquid}\Bigg] - \Bigg[\parbox{3cm}{Rate of heat form outgoing liquid}\Bigg] + \Bigg[\parbox{3cm}{Rate of heat into tank from heater}\Bigg] = \Bigg[\parbox{3cm}{Rate of heat accumulation in tank}\Bigg]
\end{align*}

Heat energy for a given liquid is given by \textbf{LAW OF WHAT}:
\begin{align}
Q = m \cdot C_p \cdot (T - T_{ref}),
\end{align}
where $Q$ is ????, $m$ is mass, $C_p$ is specific heat capacity, $T$ is temperature, and $T_{ref}$ is a reference temperature. Rates of heat into the tank due the feed can be found from derivative of (1) with respect to time.
\begin{align}
q_{in} = \frac{dm_{in}}{dt} \cdot C_p \cdot (T_i - T_{ref}),
\end{align}
where $T_i$ is the input temperature from the feed.

Similarly, we find the rate of heat out of the tank from the output as:
\begin{align}
q_{in} = \frac{dm_{out}}{dt} \cdot C_p \cdot (T - T_{ref})
\end{align}


%----------------------------------------------------------------------------------------
%	SECTION 3
%----------------------------------------------------------------------------------------

\section{Discussion}






%----------------------------------------------------------------------------------------
%	SECTION 3
%----------------------------------------------------------------------------------------

\section{Conclusion}




\end{document}